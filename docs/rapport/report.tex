\documentclass{article}

\usepackage[T1]{fontenc}
\usepackage[utf8]{inputenc}
\usepackage[french,english]{babel}

%% This package is necessary to use \includegraphics.
\usepackage{graphicx}

%% This package is necessary to define hyperlinks.
\usepackage{hyperref}

%% These packages are necessary to include code.
\usepackage{listings}
\usepackage{minted} % colored

% %% This package is needed to enchance mathematical formulas.
% \usepackage{amsmath}



\title{Intermediate report for Projet long\\ Due date February 28, 2023}
% \subtitle{fun-optimizer : optimizations for the Core language from GHC}

\author{Laure Runser, Maryline Zhang}

\begin{document}
\maketitle

\selectlanguage{english}

\section{Introduction}
This project is being supervised by Pierre-Evariste Dagand.


We didn't include in this report the formal syntax and systems we wrote.
They can be found in the git repo, under {\tt docs/syntax}, and
can be compiled into a pdf with the command {\tt make}.
If you're reading this before we merged into {\tt master}, the files are on the 
{\tt beta\_reduction} branch.


\section{Presentation}
In this project, we are looking to implement some simplifications for a fragment of 
the laguage Core, from the Haskell Glasgow Compiler. The part of Core we are 
considering form a purely functional programming language, that contain 
lamdba-calculus and polymorphic types.

You can find a description of the syntax in the file {\tt syntax.tex}.

We need to implement simplifications to make the programs more efficient to run.

We want to implement 3 types of simplifications:
\begin{enumerate}
  \item beta-reduction 
  \item inlining 
  \item let-floating
\end{enumerate}

\paragraph{Beta-reduction.}Beta-reduction is a very common type of optimization.
When an elimination form is put next to an introduction form, they sort of cancel 
each other out, and we can make a shorter, simpler block of code.

In this project, we are not implementing the actual beta-reduction, which reduces a
term completely. We are instead implementing only 3 rules:
\begin{enumerate}
  \item application of a function: {\tt (fun (x:X) -> x + 3) 4} becomes {\tt 4 + 3}
  \item application of a type function: {\tt (fun [X] = t) Y} becomes {\tt t[X:=Y]}
  \item simplification of if branches: {\tt if true then e1 else e2} becomes {e1}
\end{enumerate}

\paragraph{Inlining.}Inlining is the process of replacing a function call with
its code.
For example, 
\inputminted{haskell}{inlining.ml} 
becomes {\tt 4 + 3}, which is a lot simpler.

\paragraph{Let-floating.}With let-floating, we are moving {\em let} statements around
so that their execution is easier on the computer.

\inputminted{haskell}{let-floating.ml}
becomes
\inputminted{haskell}{let-floating2.ml}

\section{What we'we done}

\paragraph{Syntax and representation.}We started our project by implementing a data structure 
for the language. We had to make a few choices in terms of representation (see Difficulties),
and it took a few iterations for the syntax to stabilize.

We also added quite a few tools to make our life easier:
\begin{enumerate}
  \item pretty-printing
  \item ppx-show to help with debugging
  \item smart constructors to make it easier to write tests
  \item equality between terms
  \item finding the free variables of a term
  \item substitutions, renaming and alpha-equivalence
\end{enumerate}

\paragraph{Typechecker.}We then implemented a typechecker so that we had a tool to make sure
our test programs made sense. The typechecker is rather straight-forward, and we didn't have any
problems coding it.

You can find the typechecking rules in the {\tt syntax.tex} file.

\paragraph{Evaluation contexts.}In order to implement simplifications later on, we needed to 
introduce the notion of {\em evaluation contexts}. You can find the syntax and typing rules 
in the {\tt syntax.tex} file. 

We've also implemented functions to plug an evaluation context with a term, which produces a
new, well-typed term.

\paragraph{Beta-reduction.}Finally, we've started the work on the 
actual simplifications. We implemented the pseudo beta-reduction, as described in the 
presentation.

We've also started to write a proof of correctness, that you can find in the {\tt syntax.tex} file.
As of today, the proof is not complete yet, but we have made a lot of progress on it.


\section{Main difficulties}
\paragraph{Project framework.}Neither of us had ever worked on a fully-fledged OCaml project before
so it took us a while to get used to the more rigid structure of files. We also had to learn how to use 
our test framework, Alcotest, and get into the habit of writing unit tests.

\paragraph{Syntax and representation.}We learnt that writing the representation of a syntax 
is not as easy as it sounds!
It turns out there are several choices to make, each with their own pros and cons. 

We especially had a hard time implementing a system that takes into account free and bound variables.
In the end, Pierre-Evariste Dagand advised us to represent type variables with De Bruijn indexes.

However, for term variables, we decided to create an {\tt Atom} module, that contains 
a name and a number (so that we can differenciate between two variables that have the same 
name but are actually different).
This implementation was quite hard to put together and test properly.

\paragraph{Evaluation contexts.}This was a new notion for us, so we struggle to understand it at first.
We also had some trouble finding the typechecking rules that applied to them.

\paragraph{Beta-reduction.}This was our first real use of evaluation contexts, and it took us 
a while to figure out how to pattern-match them against a term so that we could simplify it.

We also had a lot of trouble writing the proof, and Pierre-Evariste was a great help for this step.

\section{What's next?}
Now that the main infrastructure of the project is done, we are going to focus on the actual simplifications.
Until now, we've implemented the beta-reduction.

Our next milestone is implementing some let-inlining, and then maybe some let-floating.

\end{document}
