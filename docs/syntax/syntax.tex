\documentclass[a4paper]{article}
\usepackage[utf8]{inputenc}
\usepackage[T1]{fontenc}
\usepackage{amsmath, amssymb}
\usepackage{pf}
\usepackage{plong}
\usepackage{listings-ocaml}
\usepackage{xcolor}
\newcommand{\TODO}[1]{\textcolor{red}{$\langle$ Pierre:}~#1\textcolor{red}{~$\rangle$}}

\begin{document}
\section{Terms}
$
\begin{array}{ccll}
\aTerm, \aTerm[1], \aTerm[2] & ::= & & \\
& | & \aBase & \text{(base)} \\
& | & \tfun \aVar \aType \aTerm & \text{(abstraction)} \\
& | & \tapp \aTerm \aBase & \text{(function application)} \\
& | & \tlet \aVar \aTerm {\aTerm[1]} & \text{(let binding)} \\
& | & \ite \aTerm {\aTerm[1]} {\aTerm[2]} & \text{(conditionnal)}\\
& | & \ttyfun \aTypeVar \aTerm & \text{(type abstraction)} \\
& | & \ttyapp \aTerm \aType & \text{(type application)} \\
& | & \ttyann \aTerm \aType & \text{(type annotation)}
\end{array}
$\\
$
\begin{array}{ccll}
\aBase & ::= & & \\
& | & \aVar & \text{(variable)} \\
& | & \true & \text{(true)} \\
& | & \false & \text{(false)} \\
\end{array}
$
\section{Types}
$
\begin{array}{ccll}
\aType, \aType[1] & ::= & & \\
& | & \aTypeVar & \text{(type variable)} \\
& | & \tyBool & \text{(bool type)} \\
& | & \aDomType \to \aCodType & \text{(function type)} \\
& | & \tforall \aTypeVar \aType & \text{(polymorphic type)} \\
& | & \aType[2] \tprod \aType[3] \tprod \ldots \tprod \aType_{k} & \text{(tuples)}
\end{array}
$
\section{Scoped terms}
\subsection{Terms environnement}

\TODO{I am not a big fan of the notation ``{\envextend a b c}'': the
  directionality of the substitution is not visually preserved. How
  about $a [ b \mapsto c ]$?}

$
\begin{array}{ccll}
\env & ::= & \\
& | & \envid & \text{(identity)}\\
& | & \envextend \rho \aVar \aBase & \text{(variable)} \\
& | & \envextend \rho \aTypeVar \aType & \text{(type variable)}
\end{array}
$\\
A scoped term is written $\scoped \aTerm \env$ and a scoped type is written $\scoped \aType \env$
\subsection{Substitution}
\subsubsection{Type substitution}

\begin{align*}
\subs \aTypeVar \envid &= \aTypeVar \\
\subs \aTypeVar {\envextend \env \aVar \aBase} &= \subs \aTypeVar \rho \\
\subs \aTypeVar {\envextend \env {\aTypeVar[1]} \aType} &=
\begin{cases}
  \aType & \text{if } \aTypeVar = \aTypeVar[1] \\
  \subs \aTypeVar \env & \text{otherwise} \\
\end{cases} \\
\subs \tyBool \env &= \tyBool \\
\subs {(\aDomType \to \aCodType)} \env &= \subs \aDomType \env \to \subs \aCodType \env \\
\subs {(\tforall \aTypeVar \aType)} \env &= \tforall {\aTypeVar[1]} {\subs \aType {\envextend \env \aTypeVar {\aTypeVar[1]}}} \quad \aTypeVar[1] \notin FV(\tforall \aTypeVar \aType) \text{ and } \aTypeVar[1] \notin FV(\env) \\
\subs {(\aType[2] \tprod \aType[3] \tprod \ldots \tprod \aType_{k})} \env &= \subs {\aType[2]} \env \tprod \subs {\aType[3]} \env \tprod \ldots \subs {\tprod \aType_{k}} \env
\end{align*}


\subsubsection{Term substitution}
$
\subs \aVar \envid = \aVar \\
\subs \aVar {\envextend \env {\aVar[1]} \aBase} =
\begin{cases}
  \aBase & \text{if } \aVar = \aVar[1] \\
  \subs \aVar \env & \text{otherwise}
\end{cases} \\
\subs \aVar {\envextend \env \aTypeVar \aType} = \subs \aVar \env \\
\subs \true \env = \true \\
\subs \false \env = \false \\
\subs {(\tfun \aVar \aType \aTerm)} \env = \tfun {\aVar[1]} {\subs \aType \env} {\subs \aTerm {\envextend \env \aVar {\aVar[1]}}} \quad \aVar[1] \notin FV(\tfun \aVar \aType \aTerm) \text{ and } \aVar[1] \notin FV(\env) \\
\subs {(\tapp \aTerm {\aTerm[1]})} \env = \tapp {\subs \aTerm \env} {\subs {\aTerm[1]} \env} \\
\subs {(\tlet \aVar \aTerm \aTerm[1])} \env = \tlet {\aVar[1]} {\subs \aTerm \env} {\subs {\aTerm[1]} {\envextend \env \aVar {\aVar[1]}}} \quad \aVar[1] \notin FV(\tlet \aVar \aTerm \aTerm[1]) \text{ and } \aVar[1] \notin FV(\env) \\
\subs {(\ite \aTerm {\aTerm[1]} {\aTerm[2]})} \env = \ite {\subs \aTerm \env} {\subs {\aTerm[1]} \env} {\subs {\aTerm[2]} \env} \\
\subs {(\ttyfun \aTypeVar \aTerm)} \env = \ttyfun {\aTypeVar[1]} {\subs \aTerm {\envextend \env \aTypeVar {\aTypeVar[1]}}} \quad \aTypeVar[1] \notin FV(\ttyfun \aTypeVar \aTerm) \text{ and } \aTypeVar[1] \notin FV(\env) \\
\subs {(\ttyapp \aTerm \aType)} \env = \ttyapp {\subs \aTerm \env} {\subs \aType \env} \\
\subs {\ttyann \aTerm \aType} \env = \ttyann {\subs \aTerm \env} {\subs \aType \env}
$
\section{Frames and stacks}
$
\begin{array}{ccll}
\aFrame & ::= & \\
& | & \aFunFrame & \text{(applied function)} \\
& | & \aPolyFrame & \text{(instantiated polymorphism)} \\
& | & \aIteFrame & \text{(if then else)}
\end{array}
$
\newline
\newline
$
\aStack ::= \emptyStack \:|\: \nonEmptyStack \aFrame \aStack \quad \text{(stacks)}
$

\TODO{Définir formellement le ``plug'' d'un terme dans une pile.}

\section{Simplifications}

\TODO{Définir ``bien scopé'' pour les terms et les piles. (Et ne pas utiliser un anglicisme, ou écrire en anglais : ``bien formé''?)}
\TODO{Problème de notation : le ``plug'' d'un terme dans une pile utilise la même notation que l'application d'une substitution. Ce sont 2 concepts distincts, il faut 2 notations différentes.}

\TODO{In the proof, explicitly point to lines of code. Here is an
  example of a pointer to a label in the OCaml code :
  line~\ref{go:fun-holefun}}

\textbf{Prop:}
Soit $\scoped \aTerm \env$ un terme bien scopé. Soit $\aStack$ une stack bien scopé.
Alors $\go {\scoped \aTerm \env} \aStack = \aTerm[1]$ avec $\stackapp \aStack {\subs \aTerm \env} \to^* \aTerm[1]$.
\begin{proof}
  \pfsketch\ Preuve par induction sur $\aTerm$
  \assume{\begin{pfenum}
            \item $\scoped \aTerm \env$ est un terme bien scopé \label{pf:term-scoped}
            \item $\aStack$ est bien scopé \label{pf:stack-scoped}
          \end{pfenum}}
  \prove{$\stackapp \aStack {\subs \aTerm \env} \to^* \go {\scoped \aTerm \env} \aStack$ }
  \step{label-1}{$\scoped \aTerm \env$ est bien scopé si les variables (termes et types) libres de $\aTerm$ sont substitués par $\env$.}
  \begin{proof}
    \pf\ Par définition d'un terme bien scopé \TODO{Cette étape semble être une tautologie, cf. hypothèse \ref{pf:term-scoped}}
  \end{proof}
  \step{label-2}{$\aStack$ est bien scopé si tous les termes ou types scopés des frames sont bien scopés.}
  \begin{proof}
    \pf\ Par définition d'une stack bien scopé \TODO{Cette étape semble être une tautologie, cf. hypothèse \ref{pf:stack-scoped}}
  \end{proof}
  \step{label-3}{\case{$\aTerm = \aVar$}}
  \begin{proof}
    \step{label-3.1}{\case{$\subs \aVar \env = \aVar[1]$}}
    \begin{proof}
      \step{label-3.1.1}{$\stackapp \aStack {\aVar[1]}$ n'est pas simplifiable}
      \begin{proof}
        \pf\ Par définition de $\to$
      \end{proof}
      \step{label-3.1.2}{$\go {\scoped \aVar \env} \aStack = \stackapp \aStack {\subs \aVar \env}$}
      \begin{proof}
        \pf\ Par définition de \textsf{go}
      \end{proof}
    \end{proof}
    \step{label-3.2}{\case{$\subs \aVar \env = \true$}}
    \begin{proof}
      \step{label-3.2.1}{Si $\aStack = \nonEmptyStack {\aIteFrame[\env']} {\aStack'}$}
      \begin{proof}
        \step{label-3.2.1.1}{$\stackapp \aStack \true \to \stackapp {\aStack'} {\subs {\aTerm[1]} {\env'}}$}
        \begin{proof}
          \pf\ $(\ite \true {\subs {\aTerm[1]} {\env'}} {\subs {\aTerm[2]} {\env'}}) \to \subs {\aTerm[1]} {\env'}$
        \end{proof}
        \step{label-3.2.1.2}{$\stackapp {\aStack'} {\subs {\aTerm[1]} {\env'}} \to^* \go {\scoped {\aTerm[1]} {\env'}} {\aStack'} $}
        \begin{proof}
          \pf\ ?
        \end{proof}
        \step{label-3.2.1.3}{$\go {\scoped \aVar \env} \aStack = \go {\scoped {\aTerm[1]} {\env'}} {\aStack'} $}
        \begin{proof}
          \pf\ Par définition de \textsf{go}
        \end{proof}
      \end{proof}
      \step{label-3.2.2}{Sinon}
      \begin{proof}
        \step{label-3.2.2.1}{$\stackapp \aStack \true$ n'est pas simplifiable}
        \begin{proof}
          \pf\ Par définition de $\to$
        \end{proof}
        \step{label-3.2.2.2}{$\go {\scoped \aVar \env} \aStack = \stackapp \aStack {\subs \aVar \env}$}
        \begin{proof}
          \pf\ Par définition de \textsf{go}
        \end{proof}
      \end{proof}
    \end{proof}
    \step{label-3.3}{\case{$\subs \aVar \env = \false$}}
    \begin{proof}
      \step{label-3.3.1}{Si $\aStack = \nonEmptyStack {\aIteFrame[\env']} {\aStack'}$}
      \begin{proof}
        \step{label-3.3.1.1}{$\stackapp \aStack \false \to \stackapp {\aStack'} {\subs {\aTerm[2]} {\env'}}$}
        \begin{proof}
          \pf\ $(\ite \false {\subs {\aTerm[1]} {\env'}} {\subs {\aTerm[2]} {\env'}}) \to \subs {\aTerm[2]} {\env'}$
        \end{proof}
        \step{label-3.3.1.2}{$\stackapp {\aStack'} {\subs {\aTerm[2]} {\env'}} \to^* \go {\scoped {\aTerm[2]} {\env'}} {\aStack'} $}
        \begin{proof}
          \pf\ ?
        \end{proof}
        \step{label-3.3.1.3}{$\go {\scoped \aVar \env} \aStack = \go {\scoped {\aTerm[2]} {\env'}} {\aStack'} $}
        \begin{proof}
          \pf\ Par définition de \textsf{go}
        \end{proof}
      \end{proof}
      \step{label-3.3.2}{Sinon}
      \begin{proof}
        \step{label-3.3.2.1}{$\stackapp \aStack \false$ n'est pas simplifiable}
        \begin{proof}
          \pf\ Par définition de $\to$
        \end{proof}
        \step{label-3.3.2.2}{$\go {\scoped \aVar \env} \aStack = \stackapp \aStack {\subs \aVar \env}$}
        \begin{proof}
          \pf\ Par définition de \textsf{go}
        \end{proof}
      \end{proof}
    \end{proof}
  \end{proof}
  \step{label-4}{\case{$\aTerm = \true$}}
  \begin{proof}
    \pf\ Même preuve que $\langle 1 \rangle 3 \langle 2 \rangle 2$ comme $\subs \true \env = \true$
  \end{proof}
  \step{label-5}{\case{$\aTerm = \false$}}
  \begin{proof}
    \pf\ Même preuve que $\langle 1 \rangle 3 \langle 2 \rangle 3$ comme $\subs \false \env = \false$
  \end{proof}
  \step{label-6}{\case{\begin{pfenum}
                        \item $\aTerm = (\tfun \aVar \aType {\aTerm[1]})$
                        \item $\forall \aStack', \forall \env'.$ si $\aStack'$ est bien scopé et $\scoped {\aTerm[1]} {\env'}$ est bien scopé, alors $\stackapp {\aStack'} {\subs {\aTerm[1]} {\env'}} \to^* \go {\scoped {\aTerm[1]} {\env'}} {\aStack'}$
                      \end{pfenum}}}
  \begin{proof}
    \step{label-6.1}{Si $\aStack = \nonEmptyStack {\aFunFrame[\env']} {\aStack'}$}
    \begin{proof}
      \step{label-6.1.1}{$\stackapp \aStack {\subs {(\tfun \aVar \aType {\aTerm[1]})} \env} \to \stackapp {\aStack'} {\subs {\aTerm[1]} {\envextend \env \aVar {\subs \aBase {\env'}}}}$}
      \begin{proof}
        \pf\ \\
        $\begin{aligned}[t]
          \tapp {\subs {(\tfun \aVar \aType {\aTerm[1]})} \env} {(\subs \aBase {\env'})}
          & = \tapp {(\tfun {\aVar[1]} {\subs \aType \env} {\subs {\aTerm[1]} {\envextend \env \aVar {\aVar[1]}}})} {(\subs \aBase {\env'})} \quad \text{avec } \aVar[1] \notin FV(t) \cup FV(\env) \\
          & \to \subs {\subs {\aTerm[1]} {\envextend \env \aVar {\aVar[1]}}} {\envextend \envid {\aVar[1]} {\subs \aBase {\env'}}} \\
          & = \subs {\aTerm[1]} {\envextend \env \aVar {\subs \aBase {\env'}}} \quad \text{comme } \aVar[1] \notin FV(\env)
        \end{aligned}$
      \end{proof}
      \step{label-6.1.2}{$\stackapp {\aStack'} {\subs {\aTerm[1]} {\envextend \env \aVar {\subs \aBase {\env'}}}} \to^* \go {\scoped {\aTerm[1]} {\envextend \env \aVar {\subs \aBase {\env'}}}} {\aStack'}$}
      \begin{proof}
        \step{label-6.1.2.1}{$\scoped {\aTerm[1]} {\envextend \env \aVar {\subs \aBase {\env'}}}$ est bien scopé}
        \begin{proof}
          \step{label-6.1.2.1.1}{$FV(u) \subseteq FV(t) \cup \{ x \}$}
          \begin{proof}
            \pf\ Par \stepref{label-6}:1 et la définition des variables libres
          \end{proof}
          \qedstep
          \begin{proof}
            \pf\ Par \stepref{label-1} et \toplevel{:1}
          \end{proof}
        \end{proof}
        \step{label-6.1.2.2}{$\aStack'$ est bien scopé}
        \begin{proof}
          \pf\ Par \toplevel{:2} et \stepref{label-2}
        \end{proof}
        \qedstep
        \begin{proof}
          \pf\ Par l'hypothèse \stepref{label-6}:2
        \end{proof}
      \end{proof}
      \step{label-6.1.3}{$\go {\scoped {(\tfun \aVar \aType {\aTerm[1]})} \env} \aStack = \go {\scoped {\aTerm[1]} {\envextend \env \aVar {\subs \aBase {\env'}}}} {\aStack'}$}
      \begin{proof}
        \pf\ Par définition de \textsf{go}
      \end{proof}
    \end{proof}
    \step{label-6.2}{Sinon}
    \begin{proof}
      \step{label-6.2.1}{$\stackapp \aStack {\subs {(\tfun \aVar \aType {\aTerm[1]})} \env} \to^* \stackapp \aStack {\tfun {\aVar[1]} {\subs \aType \env} {\aTerm[2]}}$ avec $\subs {\aTerm[1]} {\envextend \env \aVar {\aVar[1]}} \to^* \aTerm[2]$}
      \begin{proof}
        \pf\ Par substitution et définition de $\to^*$
      \end{proof}
      \step{label-6.2.2}{$\subs {\aTerm[1]} {\envextend \env \aVar {\aVar[1]}} \to^* \go {\scoped {\aTerm[1]} {\envextend \env \aVar {\aVar[1]}}} \emptyStack $}
      \begin{proof}
        \step{label-6.2.2.1}{$\scoped {\aTerm[1]} {\envextend \env \aVar {\aVar[1]}}$ est bien scopé}
        \begin{proof}
          \step{label-6.2.2.1.1}{$FV(u) \subseteq FV(t) \cup \{ x \}$}
          \begin{proof}
            \pf\ Par \stepref{label-6}:1 et la définition des variables libres
          \end{proof}
          \qedstep
          \begin{proof}
            \pf\ Par \stepref{label-1} et \toplevel{:1}
          \end{proof}
        \end{proof}
        \step{label-6.2.2.2}{$\emptyStack$ est bien scopé}
        \begin{proof}
          \pf\ Par \stepref{label-2}
        \end{proof}
        \qedstep
        \begin{proof}
          \pf\ Par l'hypothèse \stepref{label-6}:2
        \end{proof}
      \end{proof}
      \step{label-6.2.3}{$\go {\scoped {(\tfun \aVar \aType {\aTerm[1]})} \env} \aStack = \stackapp \aStack {\tfun {\aVar[1]} {\subs \aType \env} {\go {\scoped {\aTerm[1]} {\envextend \env \aVar {\aVar[1]}}} \emptyStack}}$}
      \begin{proof}
        \pf\ Par définition de \textsf{go}
      \end{proof}
    \end{proof}
  \end{proof}
  \step{label-7}{\case{\begin{pfenum}
                  \item $\aTerm = \tapp {\aTerm[1]} {\aBase}$
                  \item $\forall \aStack', \forall \env'.$ si $\aStack'$ est bien scopé et $\scoped {\aTerm[1]} {\env'}$ est bien scopé, alors $\stackapp {\aStack'} {\subs {\aTerm[1]} {\env'}} \to^* \go {\scoped {\aTerm[1]} {\env'}} {\aStack'}$
                \end{pfenum}}}
  \begin{proof}
    \step{label-7.1}{$\stackapp \aStack {\subs {(\tapp {\aTerm[1]} {\aBase})} \env} = \stackapp {(\nonEmptyStack \aFunFrame \aStack)} {\subs {\aTerm[1]} \env}$}
    \begin{proof}
      \pf\ Par définition de la substitution et du context applied function
    \end{proof}
    \step{label-7.2}{$\stackapp {(\nonEmptyStack \aFunFrame \aStack)} {\subs {\aTerm[1]} \env} \to^* \go {\scoped {\aTerm[1]} \env} {(\nonEmptyStack \aFunFrame \aStack)}$}
    \begin{proof}
      \step{label-7.2.1}{$\scoped {\aTerm[1]} \env$ et $\scoped \aBase \env$ sont bien scopés}
      \begin{proof}
        \step{label-7.2.1.1}{$FV(u) \subseteq FV(t)$ et $FV(\aBase) \subseteq FV(t)$}
        \begin{proof}
          \pf\ Par \stepref{label-7}:1 et la définition des variables libres
        \end{proof}
        \qedstep
        \begin{proof}
          \pf\ Par \stepref{label-1} et \toplevel{:1}
        \end{proof}
      \end{proof}
      \step{label-7.2.2}{$(\nonEmptyStack \aFunFrame \aStack)$ est bien scopé}
      \begin{proof}
        \pf\ Par \stepref{label-7.2.1}, \toplevel{:2} et \stepref{label-2}
      \end{proof}
      \qedstep
      \begin{proof}
        \pf\ Par l'hypothèse \stepref{label-7}:2
      \end{proof}
    \end{proof}
    \step{label-7.3}{$\go {\scoped {(\tapp {\aTerm[1]} \aBase)} \env} \aStack = \go {\scoped {\aTerm[1]} \env {(\nonEmptyStack \aFunFrame \aStack)}}$}
    \begin{proof}
      \pf\ Par définition de \textsf{go}
    \end{proof}
  \end{proof}
  \step{label-8}{\case{\begin{pfenum}
                  \item $\aTerm = \tlet \aVar {\aTerm[1]} {\aTerm[2]}$
                  \item $\forall \aStack', \forall \env'.$ si $\aStack'$ est bien scopé et $\scoped {\aTerm[1]} {\env'}$ est bien scopé, alors $\stackapp {\aStack'} {\subs {\aTerm[1]} {\env'}} \to^* \go {\scoped {\aTerm[1]} {\env'}} {\aStack'}$
                  \item $\forall \aStack', \forall \env'.$ si $\aStack'$ est bien scopé et $\scoped {\aTerm[2]} {\env'}$ est bien scopé, alors $\stackapp {\aStack'} {\subs {\aTerm[2]} {\env'}} \to^* \go {\scoped {\aTerm[2]} {\env'}} {\aStack'}$
                \end{pfenum}}}
  \begin{proof}
    \step{label-8.1}{$\stackapp \aStack {\subs {(\tlet \aVar {\aTerm[1]} {\aTerm[2]})} \env} \to^* \stackapp \aStack {(\tlet {\aVar[1]} {\aTerm[1]'} {\aTerm[2]'})}$ avec $\subs {\aTerm[1]} \env \to^* {\aTerm[1]'}$ et $\subs {\aTerm[2]} {\envextend \env \aVar {\aVar[1]}} \to^* {\aTerm[2]'}$}
    \step{label-8.2}{$\subs {\aTerm[1]} \env \to^* \go {\scoped {\aTerm[1]} \env} \emptyStack $}
    \step{label-8.3}{$\subs {\aTerm[2]} {\envextend \env \aVar {\aVar[1]}} \to^* \go {\scoped {\aTerm[2]} {\envextend \env \aVar {\aVar[1]}}} \emptyStack $}
    \step{label-8.4}{$\go {\scoped {(\tlet \aVar {\aTerm[1]} {\aTerm[2]})} \env} \aStack = \stackapp \aStack {(\tlet {\aVar[1]} {\go {\scoped {\aTerm[1]} \env} \emptyStack} {\go {\scoped {\aTerm[2]} {\envextend \env \aVar {\aVar[1]}}} \emptyStack})}$}
  \end{proof}
  \step{label-9}{\case{\begin{pfenum}
                  \item $\aTerm = \ite {\aTerm[3]} {\aTerm[1]} {\aTerm[2]}$
                  \item $\forall \aStack', \forall \env'.$ si $\aStack'$ est bien scopé et $\scoped {\aTerm[3]} {\env'}$ est bien scopé, alors $\stackapp {\aStack'} {\subs {\aTerm[3]} {\env'}} \to^* \go {\scoped {\aTerm[3]} {\env'}} {\aStack'}$
                  \item $\forall \aStack', \forall \env'.$ si $\aStack'$ est bien scopé et $\scoped {\aTerm[1]} {\env'}$ est bien scopé, alors $\stackapp {\aStack'} {\subs {\aTerm[1]} {\env'}} \to^* \go {\scoped {\aTerm[1]} {\env'}} {\aStack'}$
                  \item $\forall \aStack', \forall \env'.$ si $\aStack'$ est bien scopé et $\scoped {\aTerm[2]} {\env'}$ est bien scopé, alors $\stackapp {\aStack'} {\subs {\aTerm[2]} {\env'}} \to^* \go {\scoped {\aTerm[2]} {\env'}} {\aStack'}$
                \end{pfenum}}}
  \begin{proof}
    \step{label-9.1}{$\stackapp \aStack {\subs {(\ite {\aTerm[3]} {\aTerm[1]} {\aTerm[2]})} \env} = \stackapp {(\nonEmptyStack \aIteFrame \aStack)} {\subs {\aTerm[3]} \env}$}
    \step{label-9.2}{$\stackapp {(\nonEmptyStack \aIteFrame \aStack)} {\subs {\aTerm[3]} \env} \to^* \go {\scoped {\aTerm[3]} \env} {(\nonEmptyStack \aIteFrame \aStack)}$}
    \step{label-9.3}{$\go {\scoped {(\ite {\aTerm[3]} {\aTerm[1]} {\aTerm[2]})} \env} \aStack = \go {\scoped {\aTerm[3]} \env {(\nonEmptyStack \aIteFrame \aStack)}}$}
  \end{proof}
  \step{label-10}{\case{\begin{pfenum}
                  \item $\aTerm = \ttyfun \aTypeVar {\aTerm[1]} $
                  \item $\forall \aStack', \forall \env'.$ si $\aStack'$ est bien scopé et $\scoped {\aTerm[1]} {\env'}$ est bien scopé, alors $\stackapp {\aStack'} {\subs {\aTerm[1]} {\env'}} \to^* \go {\scoped {\aTerm[1]} {\env'}} {\aStack'}$
                \end{pfenum}}}
  \begin{proof}
    \step{label-10.1}{Si $\aStack = \nonEmptyStack {\aPolyFrame[\env']} {\aStack'}$}
    \begin{proof}
      \step{label-10.1.1}{$\stackapp \aStack {\subs {(\ttyfun \aTypeVar {\aTerm[1]})} \env} \to \stackapp {\aStack'} {\subs {\aTerm[1]} {\envextend \env \aTypeVar {\subs {\aType[1]} {\env'}}}}$}
      \step{label-10.1.2}{$\stackapp {\aStack'} {\subs {\aTerm[1]} {\envextend \env \aTypeVar {\subs {\aType[1]} {\env'}}}} \to^* \go {\scoped {\aTerm[1]} {\envextend \env \aTypeVar {\subs {\aType[1]} {\env'}}}} {\aStack'}$}
      \step{label-10.1.3}{$\go {\scoped {(\ttyfun \aTypeVar {\aTerm[1]})} \env} \aStack = \go {\scoped {\aTerm[1]} {\envextend \env \aTypeVar {\subs {\aType[1]} {\env'}}}} {\aStack'}$}
    \end{proof}
    \step{label-10.2}{Sinon}
    \begin{proof}
      \step{label-6.2.1}{$\stackapp \aStack {\subs {(\ttyfun \aTypeVar {\aTerm[1]})} \env} \to^* \stackapp \aStack {\ttyfun {\aTypeVar[1]} {\aTerm[2]}}$ avec $\subs {\aTerm[1]} {\envextend \env \aTypeVar {\aTypeVar[1]}} \to^* \aTerm[2]$}
      \step{label-6.2.2}{$\subs {\aTerm[1]} {\envextend \env \aTypeVar {\aTypeVar[1]}} \to^* \go {\scoped {\aTerm[1]} {\envextend \env \aTypeVar {\aTypeVar[1]}}} \emptyStack $}
      \step{label-6.2.3}{$\go {\scoped {(\ttyfun \aTypeVar {\aTerm[1]})} \env} \aStack = \stackapp \aStack {\ttyfun {\aTypeVar[1]} {\go {\scoped {\aTerm[1]} {\envextend \env \aTypeVar {\aTypeVar[1]}}} \emptyStack}}$}
    \end{proof}
  \end{proof}
  \step{label-11}{\case{\begin{pfenum}
                  \item $\aTerm = \ttyapp {\aTerm[1]} {\aType[1]}$
                  \item $\forall \aStack', \forall \env'.$ si $\aStack'$ est bien scopé et $\scoped {\aTerm[1]} {\env'}$ est bien scopé, alors $\stackapp {\aStack'} {\subs {\aTerm[1]} {\env'}} \to^* \go {\scoped {\aTerm[1]} {\env'}} {\aStack'}$
                \end{pfenum}}}
  \begin{proof}
    \step{label-11.1}{$\stackapp \aStack {\subs {(\ttyapp {\aTerm[1]} {\aType[1]})} \env} = \stackapp {(\nonEmptyStack \aPolyFrame \aStack)} {\subs {\aTerm[1]} \env}$}
    \step{label-11.2}{$\stackapp {(\nonEmptyStack \aPolyFrame \aStack)} {\subs {\aTerm[1]} \env} \to^* \go {\scoped {\aTerm[1]} \env} {(\nonEmptyStack \aPolyFrame \aStack)}$}
    \step{label-11.3}{$\go {\scoped {(\ttyapp {\aTerm[1]} {\aType[1]})} \env} \aStack = \go {\scoped {\aTerm[1]} \env {(\nonEmptyStack \aPolyFrame \aStack)}}$}
  \end{proof}
  \step{label-12}{\case{\begin{pfenum}
                  \item $\aTerm = \ttyann {\aTerm[1]} \aType$
                  \item $\forall \aStack', \forall \env'.$ si $\aStack'$ est bien scopé et $\scoped {\aTerm[1]} {\env'}$ est bien scopé, alors $\stackapp {\aStack'} {\subs {\aTerm[1]} {\env'}} \to^* \go {\scoped {\aTerm[1]} {\env'}} {\aStack'}$
                \end{pfenum}}}
  \begin{proof}
    \step{label-12.1}{$\stackapp \aStack {\subs {\ttyann {\aTerm[1]} \aType} \env} \to \stackapp \aStack {\subs {\aTerm[1]} \env}$}
    \step{label-12.2}{$\stackapp \aStack {\subs {\aTerm[1]} \env} \to^* \go {\scoped {\aTerm[1]} \env} \aStack$}
    \step{label-12.3}{$\go {\scoped {\ttyann {\aTerm[1]} \aType} \env} \aStack = \go {\scoped {\aTerm[1]} \env} \aStack$}
  \end{proof}
\end{proof}

\begin{figure}
  \ocamlfrom{stack.ml}{go}
\end{figure}

\end{document}
